\documentclass[a4paper]{article}
%\usepackage[letterpaper, margin=1.5in]{geometry}

\usepackage{fontspec}
\usepackage{mathpazo}
\setmainfont
     [ BoldFont       = texgyrepagella-bold.otf ,
       ItalicFont     = texgyrepagella-italic.otf ,
       BoldItalicFont = texgyrepagella-bolditalic.otf ]
     {texgyrepagella-regular.otf}
\setmainfont{Gill Sans MT}

\usepackage[english]{babel}
\usepackage[utf8]{inputenc}
\usepackage{amsmath}
\usepackage{graphicx}
\usepackage[colorinlistoftodos]{todonotes}
\usepackage{physics}

\title{Previous Work}

\author{David McPherson}

\date{\today}

\begin{document}

\cite{KeneJaime}

\section{Previous Work}
Just as deterministic reachability is intertwined with optimal control, stochastic reachability is intertwined with stochastic optimal control.
Work like Bertsekas and Shreve \cite{Bertsekas} lays a solid foundation for analyzing stochastic optimal control with discrete time.
Abate et. al \cite{AbateStoch} picks up these tools for stochastic optimal control and applies it to the hybrid reachability control problem.
We will carry the mantle of this research a little further by starting to peer at the particularly useful problem described in our Motivation.

To do so, we must analyze continuous time evolution of a stochastic system with Gaussian noise.
Here we tap into the venerable heritage of probability theory and stochastic calculus.
The study of stochastic differential systems is immense.
Here we lean heavily on the insights afforded in Evans' textbook on SDEs\cite{EvansSDE}.

All this analysis is towards the goal of molding this problem into a form acceptable to the powerful Level Set Toolbox created by Mitchell \cite{MitchellToolbox}.
By casting our problem as a Hamilton-Jacobi partial differential equation, we can leverage advances in solving these uniquely useful equations.
It is worth noting that Mitchell \cite{MitchellToolbox} actually contains an example of a stochastic differential system with white Gaussian noise, just as we're investigating.
This work expounds upon the formulation Mitchell obtained.
Where Mitchell simply pointed to Evans \cite{EvansSDE} and left out the derivation for his formula, we provide a rough derivation.
We also apply this formulation to reachability, which Mitchell did not.

  % =========== References =========== %
\section{Bibliography}

  \bibliographystyle{plain}
  \bibliography{primary_cite}

\end{document}

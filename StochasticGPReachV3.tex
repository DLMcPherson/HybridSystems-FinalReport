\documentclass[a4paper]{article}

\usepackage{fontspec}
\usepackage{mathpazo}
\setmainfont
     [ BoldFont       = texgyrepagella-bold.otf ,
       ItalicFont     = texgyrepagella-italic.otf ,
       BoldItalicFont = texgyrepagella-bolditalic.otf ]
     {texgyrepagella-regular.otf}
\setmainfont{Gill Sans MT}

\usepackage[english]{babel}
\usepackage[utf8]{inputenc}
\usepackage{amsmath}
\usepackage{graphicx}
\usepackage[colorinlistoftodos]{todonotes}
\usepackage{physics}

%\DeclareMathOperator{\Tr}{Tr}

\title{Stochastic Reachability with Gaussian Noise}

\author{David McPherson}

\date{\today}

\begin{document}
\maketitle

\section{Introduction}
Reachability has become a valuable tool for analyzing the safety of complex dynamic systems.
Classic reachability formulates the disturbance as playing a dynamic game, wherein the disturbance aggressively chooses the worst possible choice to push the control to its limits.
By working in this worst-case scenario, we can effectively certify safety.
However, such worst case estimates for safety are often too conservative.
The agressive disturbance always fills the void for some uncertain parameter that we at least have bounds on.
However, in many cases we have more than just bounds on the disturbance.
Oftentimes we have some statistical information on the distribution of the disturbance.
Rather than throw away this distributional information, it would be ideal if we could incorporate it.

This is the idea behind stochastic reachability.
Instead of providing 100\% guarantees that some region is safe, it provides likelihoods of that inital state staying safe.
It effectively mixes in the doubt inherent in the disturbance's uncertainty, without making assumptions about the worst-case.

The downside is that stochastic analysis is complex.
Good computational methods for analyzing the differential games for classic reachability exist.
However, the computational methods needed for analyzing stochastic differential equations with disturbances drawn from arbitrary distributions just aren't there yet.

This work is an introduction to the world of stochastic reachability, and allowed me to begin sampling the plethora of topics required to make stochastic analysis work.
Instead of analyze reachability for arbitrary probabilistic disturbances, I focus on disturbances that are normally distributed.
Although this is a very narrow subset of all stochastic reachability problems, I feel it is the most interesting narrow subset.

\section{Problem Statement and Motivation}
Control affine form

Jaime and Kene

\section{Previous Work}

\section{Derivations}

\section{Applications}
\subsection{Quadrotor Safety with Learned Disturbances}

\section{Conclusion}


\end{document}
